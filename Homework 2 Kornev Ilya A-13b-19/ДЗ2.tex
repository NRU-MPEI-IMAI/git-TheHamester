\documentclass[12pt]{article}
\usepackage[utf8]{inputenc}
\usepackage[russian]{babel}
\usepackage{amsmath}
\usepackage[pdf]{graphviz}
\usepackage{morewrites}
\usepackage{amssymb}

\begin{document}
	\title{Домашняя работа по ТМВ №2}
	\author{Корнев Илья А-13б-19}
	\date{}
	\maketitle
	\setlength{\footskip}{60pt}
	
	\section*{Упражнение 1.}
	Построим КС-грамматику для следующих языков:\\
	$\boldsymbol{\indent1.\  L=\{\omega \in \{a,b,c\}^* \vert \omega \textup{ содержит подстроку } aa\}}$
	$$S\ :=\ AaaA$$
	$$A\ :=\ aA\ \vert\ bA\ \vert\ cA\ \vert\ \lambda$$
	
	$\boldsymbol{2.\  L=\{\omega \in \{a,b,c\}^* \vert \omega \neq \omega^R\}}$
	$$S\ \to\ aSa\ \vert\ bSb\ \vert\ cSc$$
	$$S\ \to\ aAb\ \vert\ aAc\ \vert\ bAa\ \vert\ bAc\ \vert\ cAa\ \vert\ cAb$$
	$$A\ \to\ aA\ \vert\ bA\ \vert\ cA \vert\ \lambda$$
	
	$\boldsymbol{3.\  L=\{ \omega \in \{ \emptyset, \mathbb{N}, '\{', '\}', \cup, ',' \}^* \omega \textup{ корректное множество} \}}$
	$$S\ \to\ A\ \cup\ S\ \vert\ A$$
	$$A\ \to\ \emptyset\ \vert\ \mathbb{N},\{B\}$$
	$$B\ \to\ SC\ \vert\ \lambda$$
	$$C\ \to\ ,S\ \vert\ \lambda$$
	
	\section*{Упражнение 2.}
	Рассмотрим язык $\boldsymbol{A = \{1^m + 1^n = 1^{m+n} \vert m,n \in \mathbb{N} \}}$:\\
	a). Докажем, что язык A не регулярный используя лемму о накачке:
	Фиксируем произвольное $n$:
	$$\omega = 1^n + 1^n = 1^{2n}$$
	$$x = 1^{n-1}$$
	$$y = 1$$
	$$z = +1^n=1^{2n}$$
	$xy^kz \notin A при k \geq 2 \implies $ язык A не регулярный.\\
	
	б). Построим КСГ для языка A:
	$$S\ \to\ A\ \vert\ +B$$
	$$A\ \to\ 1A1\ \vert\ +B$$
	$$B\ \to\ 1B1\ \vert\ =$$
	
	\section*{Упражнение 3.}
	$h\ - $ движение хозяина на 1 шаг вперед.\\
	$d\ - $ движение собаки на 1 шаг вперед.\\
	Прогулка может считаться оконченной когда собака и хозяин оказываются в одном месте.\\
	Поводок ограничивает движение на 2 шага от друг друга.\\
	$\boldsymbol{1.\ D_1 = \{ \omega \in \{ h, d \}^* \vert \omega \textup{ описывает последовательность ваших шагов и шагов вашей}}$ \\
	$\boldsymbol{\textup{собаки с поводком} \}}$\\
	a). Покажем, что язык $D_1$ регулярный построив ДКА, распознающий его:
	\begin{center}
		\digraph{dot/p3}
	\end{center}
	б). Построим КСГ для языка $D_1$:
	$$S\ \to\  dA\ \vert\ hB$$
	$$A\ \to\ d\ \vert\ dC$$
	$$B\ \to\ d\ \vert\ hD$$
	$$C\ \to\ hA$$
	$$D\ \to\ dB$$
	
		$\boldsymbol{2.\ D_2 = \{ \omega \in \{ h, d \}^* \vert \omega \textup{ описывает последовательность ваших шагов и шагов вашей}}$ \\
	$\boldsymbol{\textup{собаки без поводка} \}}$
	$$D_2 = \{\omega \in \{ h, d \}^* \vert |\omega|_h = |\omega|_d \}$$
	a). Докажем что язык $D_2$ не регулярный, используя лемму о накачке:\\
	Фиксируем произвольное $n$:
	$$\omega = h^nd^n$$
	$$x = h^{n-1}$$
	$$y = h$$
	$$z = d^n$$
	$xy^kz \notin A при k \geq 2 \implies$ язык $D_2$ не регулярный.\\
	
	б). Построим КСГ для языка $D_2$:
	$$S\ \to\ SS\ \vert\ hSd\ \vert\ dSh\ \vert\ \lambda$$
	
	\section*{Упражнение 4.}
	
	$Perm(\omega)\ -$ множество всех пурметаций строки $\omega$.\\
	$Perm(L)\ -$ объединение $Perm(\omega)$ для всех $\omega \in L$\\
	Рассмотрим следующие языки охарактеризованными регулярными выражениями:\\
	
	$\boldsymbol{1.\ R_1 = \{01\}^*}$\\
	$\boldsymbol{Perm(R_1) = \{\omega \in \{ 0, 1 \} \vert |\omega|_0 = |\omega|_1 \}}$\\
	Докажем что язык $Perm(R_1)$ не регулярный, используя лемму о накачке:\\
	Фиксируем произвольное $n$:
	$$\omega = 0^n1^n$$
	$$x = 0^{n-1}$$
	$$y = 0$$
	$$z = 1^n$$
	$xy^kz \notin A при k \geq 2 \implies$ язык $Perm(R_1)$ не регулярный.\\
	Построим КСГ для языка $Perm(R_1)$ и покажем, что он контекстно-свободный:
	$$S\ \to\ SS\ \vert\ 0S1\ \vert\ 1S0\ \vert\ \lambda$$
	 
	 $\boldsymbol{2.\ R_2 = \{0^* + 1^*\} = \{ 0^n \vert n \geq 0 \} \cup \{ 1^n \vert n \geq 0 \}}$\\
	 $\boldsymbol{Perm(R_2) = R_2}$\\
	 Покажем, что язык $Perm(R_2)$ регулярный построив ДКА, распознающий его:
	 \begin{center}
		 \digraph{dot/p4}
	 \end{center}
	 Из этого следует, что язык $Perm(R_2)$ также является контекстно-свободным.\\
	 
	 $\boldsymbol{3.\ R_3 = {012}^*}$\\
	 $\boldsymbol{Perm(R_3) = \{\omega \in \{ 0, 1, 2 \} \vert |\omega|_0 = |\omega|_1 = |\omega|_2 \}}$\\
	 Покажем, что язык $Perm(R_3)$ не является контекстно-свободным, используя лемму о накачке для контектсно-свободных языков:\\
	 Фиксируем произвольное $n$:
	 $$\omega = a^nb^nc^n=uvwxy,\ |vx| \geq 1,\ |vwx| \leq n$$
	 $$u = a^{n-1}$$
	 $$v = a$$
	 $$w = b^{n-1}$$
	 $$x = b$$
	 $$y=c^n$$
	 $uv^kwx^ky \notin Perm(R_3) при k \geq 2 \implies$ язык $Perm(R_3)$ не контекстно-свободный, а значит тем более не регулярный.\\ 
	 
	 \section{Упражнение 5.}
	 При условии, что все правила КС-грамматики имеют форму:
	 $$A\ \to\ \lambda$$
	 $$A\ \to\ B$$
	 $$A\ \to\ aB$$
	 Возможно построить эквивалентный ей НКА, следующим образом.\\
	 1). Пусть мы имеем КС-грамматику $G = (V, \Sigma_G, R, S)$\\
	 \indent$V\ - $ множество нетерминальных символов\\
	 \indent$\Sigma_G\ - $ множество терминальных символов\\
	 \indent$R = \{ (v,\omega)\ \vert\ v \in V,\ \omega \in (\{ \lambda \} \cup \Sigma_G \cup (\Sigma_G \cdot V)) \}\ -$ множество правил праволинейной КС-грамматикиб где $\Sigma_G \cdot V = \{ cv\ \vert\ c \in \Sigma_G,\ v \in V \}\ -$ конкатенация множеств\\
	 \indent$S\ -$ аксиома\\
	 Построим НКА $N = (\Sigma_N, Q, s, T, \delta)$, такой что:
	 $$\Sigma_N = \Sigma_G$$
	 $$Q = V$$
	 $$s = S$$
	 $$T = \{ v\ \vert\ \exists (v,\lambda) \in R \}$$
	 Функции переходов определяются следующим образом:
	 $$\delta(q,c)=\{ v\ \vert\ \exists (v,cq) \in R,\ q \in V \}$$
	 
	 Продемонстрируем работу алгоритма:\\
	 Имеем КС-грамматику:\\
	 $$A\ \to\ aB\ \vert\ bC$$
	 $$B\ \to\ aB\ \vert\ \lambda$$
	 $$C\ \to\ aD\ \vert\ A\ \vert\ bC$$
	 $$D\ \to\ aD\ \vert\ bD\ \vert\ \lambda$$
	 
	 По нашему алгоритму получаем следующий автомат:\\
	 \begin{center}
		\digraph{dot/p51}
	 \end{center}
	
	 Докажем от противного, что автомат $N$ может получить только слова из языка грамматики $G$.\\
	 \indent Язык грамматики $G$: $L(G) = \{ \omega \in \Sigma_G\ \vert\ S \Rightarrow^* \omega \}$\\
	 \indent Язык автомата $N$: $L(N) = \{ \omega\ \vert\ \delta^*(s,\omega) \in T \}\ -$ множество слов, которые допускает автомат $N$, где $\delta^*(q,\omega)$ рекурсивно определен следующим образом:\\
	 $$\delta^*(q,\lambda) = q$$
	 $$\delta^*(q,\omega c) = v \in \delta(\delta^*(q,\omega), c)$$
	 Пусть $\exists \omega \in L(N)\ \vert\ \omega \notin L(G)$\\
	 В этом случае, в какойто момент при при продолжительных переходах мы можем прийти к состоянию, когда:
	 $$\delta^n(s,\omega c)=v_1,  \textup{ но } S \Rightarrow^n \omega cv_2$$ 
	 $$v_1 \in \delta(\delta^{n-1}(s,\omega),c),\ v_2 \notin \delta(\delta^{n-1}(s,\omega),c)$$
	 Однако $\forall q \in V = Q\ \delta(q,c) = \{ v\ \vert\ \exists(v,cq) \in R \}\ -$ противоречие\\
	 
	 2). Теперь пусть мы имеем НКА $N = (\Sigma_N, Q, s, T, \delta)$\\
	 Возможно построить эквивалентную КС-грамматику $G = (V, \Sigma_G, R, S)$ следующим образом:\\
	 $$V=Q$$
	 $$\Sigma_G = \Sigma_N$$
	 $$S = s$$
	 $$R = \{(v,\omega)\ \vert\ v \in Q,\ \omega = cq,\ q \in \delta(v,c) \neq \emptyset  \} \cup \{ (v,\lambda)\ \vert\ v \in T \}$$
	 
	 Продемонстрируем работу алгоритма:\\
	 Имеем НКА:\\
	 \begin{center}
		\digraph{dot/p52}
	 \end{center}
	 
	 По нашему алгоритму получаем следующую КС-грамматику:\\
	 $$A\ \to\ aA\ \vert\ aB\ \vert\ D$$
	 $$B\ \to\ aB\ \vert\ bE\ \vert\ aC\ \vert\ C$$
	 $$C\ \to\ aC\ \vert\ bC\ \vert\ aF\ \vert\ \lambda$$
	 $$D\ \to\ bA\ \vert\ \lambda$$
	 $$E \to\ F\ \vert\ \lambda$$
	 $$F\ \to\ aF\ \vert\ bC$$
	 
	 Докажем от противного, что КС-грамматика $G$ может получить только слова из языка НКА $N$.\\
	 \indent Язык грамматики $G$: $L(G) = \{ \omega \in \Sigma_G\ \vert\ S \Rightarrow^* \omega \}$\\
	 \indent Язык автомата $N$: $L(N) = \{ \omega\ \vert\ \delta^*(s,\omega) \in T \}\ -$ множество слов, которые допускает автомат $N$, где $\delta^*(q,\omega)$ рекурсивно определен следующим образом:\\
	 $$\delta^*(q,\lambda) = q$$
	 $$\delta^*(q,\omega c) = v \in \delta(\delta^*(q,\omega), c)$$
	 Пусть $\exists \omega \in L(G)\ \vert\ \omega \notin L(N)$\\
	 В этом случае, в какойто момент при при продолжительных приложениях правил G мы можем прийти к состоянию, когда:
	 $$S \Rightarrow^n \omega cv_1,  \textup{ но } \delta^n(s,\omega c) = \delta(\delta^{n-1}(s,\omega),c) = v_2$$ 
	 $$(v_1,c) \in R, \textup{ но } (v_2,c) \notin R$$
	 Однако $\forall q \in \delta(v,c)\ (v,cq) \in R\ -$ противоречие.
\end{document}